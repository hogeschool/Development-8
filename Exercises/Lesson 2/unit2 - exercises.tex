\documentclass[]{article}

\usepackage{enumitem}
\usepackage{geometry}[top = 1cm, bottom = 2cm, left = 1.5cm, right = 1.5cm]
\usepackage{listings}

%opening
\title{Development 8 - Exercises\\Unit 2}
\author{}
\date{}

\newcounter{ExerciseCount}
\setcounter{ExerciseCount}{1}

\newcommand{\functionEx}[3]{
  Implement a function\\\\
   \texttt{let #1 = #2}\\\\ #3
}

\newcommand{\exercise}[1]{\noindent \textbf{Exercise \theExerciseCount:}\\\\ #1 \addtocounter{ExerciseCount}{1}
}

\lstset{
  breaklines = true,
  basicstyle = \ttfamily,
  tabsize = 2
}

\begin{document}
\maketitle

\noindent
For these exercises use the polymorphic definition of \texttt{List<a>} using discriminated unions seen in class:

\begin{lstlisting}
type List<a> = {
  kind: "empty"
} | {
  kind: "cons"
  head: a
  tail: List<a>
}
\end{lstlisting}

\exercise{
  \functionEx{last}{<a>(l: List<a>): a}{
    that returns the last element of a list.
  }
}\\

\exercise{
  \functionEx{rev}{<a>(l: List<a>): List<a>}{
    that creates a list with the elements of \texttt{l} in reverse order.
  }
}\\

\exercise{
  \functionEx{append}{<a>(l1: List<a>) => (l2: List<a>):\\  List<a>}{
    that adds all the elements of \texttt{l2} after those in \texttt{l1}.
  }
}\\

\exercise{
  \functionEx{nth}{<a>(n: number) => (l: List<a>): a}{
    that returns the element in position \texttt{n} in \texttt{l}.
  }
}\\

\exercise{
  \functionEx{palindrome}{<a>(l: List<a>): boolean}{
    that checks if a list is palindrome. A list is palindrome if it is equal to its inverse.
  }
}\\

\exercise{
  \functionEx{compress}{<a>(l: List<a>): List<a>}{
    that removes consecutive occurrences of the same element in the list. For example \texttt{compress [a;a;a;a;b;b;c;c;b] = [a;b;c;b]}.
  }
}\\

\exercise{
  \functionEx{caesarCypher}{(l: List<string>) => (shift: number):\\ List<string>}{
    The Caesar's cypher take a text, represented as a list of characters (note that Typescript does not support the type \texttt{char} so you can use a list of \texttt{string} with only one character), and shifts all the letters (so only if the character is an alphabetical character) in it up by the number of position specified by \texttt{shift}. If the letter goes past \texttt{z} it restarts from \texttt{a}. You can assume that all the text is in lower-case letter. For instance:\\\\    
    \texttt{shift("c")(5) = h}\\
    \texttt{shift("y")(5) = d}\\\\
    The ASCII code for a specific character in a \texttt{string} can be obtained by using the method \texttt{charCodeAt} that takes as input the position of the character of the string you want to get the ASCII code of. For instance:\\\\
    \texttt{"Caesar".charCodeAt(2) = 101}\\\\
    \textbf{Advanced: } Try to support also upper-case letters in the text.
  }
}\\

\end{document}
