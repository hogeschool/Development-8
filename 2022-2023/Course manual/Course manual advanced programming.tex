\documentclass[titlepage, a4paper, openbib, 10pt]{article}

%#####################################
%Usepackages en installingen
\usepackage[top=1in, bottom=1in, left=1in, right=1in]{geometry}
\usepackage[pdftex]{graphicx}
\usepackage{fancyhdr}
\usepackage{sectionbox}
\usepackage[english]{babel}
\usepackage{chngcntr}
\usepackage{cite}
\usepackage{comment}
\usepackage{url}
\usepackage{makeidx}
\usepackage{paralist}
\usepackage{enumitem}
\usepackage{tocloft}
\usepackage{listliketab}	
\usepackage[table]{xcolor}
\usepackage{tabularx}
\usepackage{epsfig}
\usepackage{pdflscape}
\usepackage{pdfpages}
\usepackage{float}
\usepackage{multirow} 
\usepackage{rotating}
\usepackage[utf8]{inputenc}
\usepackage{color}
\usepackage{fp}
\usepackage[hidelinks]{hyperref}
\hypersetup{
    colorlinks=false,
    linkcolor=black,
    filecolor=black,
    urlcolor=black,
}
%\usepackage{draftwatermark}
%\SetWatermarkText{\textsc{Draft}}
%\SetWatermarkScale{5}
\newcommand{\red}[1]{
\textcolor{red}{#1}
}
\usepackage{listings}
\lstset{language=C,
basicstyle=\ttfamily\footnotesize,
frame=shadowbox,
escapeinside={(*@}{@*)},
mathescape=true,
showstringspaces=false,
showspaces=false,
breaklines=true}


%#####################################
%Glossary
%\usepackage{glossaries}
\usepackage{mfirstuc}
\usepackage[toc,nonumberlist,nopostdot]{glossaries}
\makeglossaries
\newglossaryentry{fpvsimp}{
      name=fpvsimp
    , description={The learning objective stating that at the end of this course: the student \textbf{understands} concepts of functional vs imperative semantics.}
    , first={\textbf{understands} the fundamental semantic difference between functional and imperative programming. \texttt{(FP VS IMP)}}
    , symbol=\texttt{FP VS IMP}
    }


\newglossaryentry{red}{
	name=red
	, description={The learning objective stating that at the end of this course: the student \textbf{understands} reduction strategies such as $\rightarrow_\beta$.}
	, first={\textbf{understands} reduction strategies such as $\rightarrow_\beta$. \texttt{(RED)}}
	, symbol=\texttt{RED}
}


\newglossaryentry{typ}{
	name=typ
	, description={The learning objective stating that at the end of this course: the student \textbf{understands} the basics of a functional type system.}
	, first={\textbf{understands} the basics of a functional type system. \texttt{(TYP)}}
, symbol=\texttt{TYP}
}


\newglossaryentry{fpext}{
	name=fpext
	, description={The learning objective stating that at the end of this course: the student \textbf{can program} with the common extensions of a functional programming language wrt the basic lambda calculus, such as \texttt{let}, \texttt{if}, \textbf{let-rec}, unions, tuples, records, etc. The language of focus is Typescript.}
	, first={\textbf{can program} with the typical constructs of a modern functional language. The language of focus is F\#. \texttt{(FP EXT)}}
, symbol=\textbf{FP EXT}
}

%\loadglsentries{Glossary}


%\usepackage{showframe} %tmp
%#####################################
%Nieuwe commando's
\newcommand{\HRule}{\rule{\linewidth}{1pt}}
\newcommand{\organisatie}{\uppercase{Hogeschool Rotterdam / CMI}}
\newcommand{\modulenaam}{Functional Programming}
\newcommand{\modulecode}{\uppercase{INFADP01-D/INFADP21-D}}
\newcommand{\studiejaar}{\uppercase{2022-2023}}
\newcommand{\stdPunten}{4}
\renewcommand{\author}{Francesco Di Giacomo, Giuseppe Maggiore}

\definecolor{lichtGrijs}{RGB}{169,169,169}



%#####################################
%Index en styling
\setlength{\cftbeforesecskip}{10pt}
\setlength\parindent{0pt}
\makeindex
\graphicspath{{../Img/}}
\counterwithin{figure}{subsection}
\pagestyle{fancy}
\setcounter{secnumdepth}{5}
\setcounter{tocdepth}{5}

%#####################################
%     Alles voor header/footer
\fancyhf[HL]{\nouppercase{\textit{\leftmark}}}
\setlength{\headheight}{36pt}
\lhead{\uppercase{\footnotesize Course description}}
\chead{\footnotesize \organisatie}
\rhead{\includegraphics[width=0.09\textwidth]{logo}}

\lfoot{\scriptsize \modulenaam}
\cfoot{\scriptsize \today}
\rfoot{\small \thepage}

\renewcommand{\headrulewidth}{0.4pt}
\renewcommand{\footrulewidth}{0.4pt}
%#####################################

\begin{document}


%#####################################
%Titlepage
\begin{titlepage}
\thispagestyle{fancy}
\input{Voorblad}
\end{titlepage}

%####### Contentpagina ########
%\renewcommand{\baselinestretch}{1.5}\normalsize
%\tableofcontents
%\newpage
%\listoffigures
%\newpage
%\listoftables
%\newpage

%########### Inhoud ###########

\shadowsectionbox
\section*{Module description}
\begin{tabularx}{\textwidth}{|>{\columncolor{lichtGrijs}} p{.26\textwidth}|X|}
	\hline
	\textbf{Module name:} & \modulenaam\\

	\hline
	\textbf{Module code: }& \modulecode\\
	\hline
	\textbf{Study points \newline and hours of effort:} & This module gives \stdPunten{}  ects, in correspondence with \FPeval{\result}{clip(\stdPunten*28)}\result{} hours:
	\begin{itemize}
		\item 2 X 7 hours of combined lecture and practical
		\item the rest is self-study
	\end{itemize} \\
	\hline
	\textbf{Examination:} & Written Exam and Practical assessment \\
	\hline
	\textbf{Course structure:} & Lectures, self-study, and practical exercises \\
	\hline
	\textbf{Prerequisite knowledge:} & all INFDEV courses. \\
	\hline
	\textbf{Learning materials:}  &
		\begin{itemize}
			\item Book: Don Syme - Expert F\#
			\item exercises and assignments, to be done at home and during the practical part of the lectures (pdf): found on N@tschool
		\end{itemize} \\
	\hline
	\textbf{Connected to competences:} & realiseren en ontwerpen \\
	\hline
	\textbf{Learning objectives:} &
		At the end of the course, the student:
			\begin{itemize}
                \item \glsfirst{fpvsimp}
                \item \glsfirst{red}
                \item \glsfirst{typ}
                \item \glsfirst{fpext}
			\end{itemize} \\
	\hline
%\end{tabularx}
%\newpage
%
%\begin{tabularx}{\textwidth}{|>{\columncolor{lichtGrijs}} p{.26\textwidth}|X|}
%	\hline
%	\textbf{Content:}&
%	\begin{itemize}
%		\item _
%	\end{itemize} \\
%	\hline
	\textbf{Course owners:} & \author\\
	\hline
	\textbf{Date:} & \today \\
	\hline
\end{tabularx}
%\newpage


\newpage
\section{General description}

Functional programming and functional programming languages are increasing in popularity for multiple reasons and in multiple ways, to the point that even mainstream languages such as Python, C++, C\#, and Java are being extended with more and more functional programming features such as tuples, lambda's, higher order functions, and even monads such as LINQ and async/await. Whole architectures such as the popular map/reduce are strongly inspired by functional programming.

``Java™ developers should learn functional paradigms now, even if they have no immediate plans to move to a functional language such as Scala or Clojure. Over time, all mainstream languages will become more functional'' [IBM].

``LISP is worth learning for a different reason — the profound enlightenment experience you will have when you finally get it. That experience will make you a better programmer for the rest of your days, even if you never actually use LISP itself a lot.'' – Eric S. Raymond

``SQL, Lisp, and Haskell are the only programming languages that I've seen where one spends more time thinking than typing.'' – Philip Greenspun

``I do not know if learning Haskell will get you a job. I know it will make you a better software developer.'' – Larry O’ Brien

The reason for this growth is to be found in the safe and deep expressive power of functional languages, which are capable of recombining simpler elements into powerful, complex other elements with less space for mistakes and more control in the hands of the programmer. This comes at a fundamental cost: functional languages are structurally different from imperative and object oriented languages, and thus a new mindset is required of the programmer that wishes to enter this new world. Moreover, functional languages often require more thought and planning, and are thus experienced, especially by beginners, as somewhat less flexible and supporting of experimentation.

\subsection{Relationship with other didactic units and required knowledge}
This module completes and perfects the understanding and knowledge of programming that was set up in the Development courses of the first year. The preliminary knowledge necessary to fully understand this course covers the following topics:

\begin{itemize}[noitemsep]
\item Semantics of programming languages and its evaluation.
\item The memory model of \textit{Stack} and \textit{Heap}.
\item \textit{Type systems} and type checking.
\item Dynamic vs Static typing.
\end{itemize}

\section{Course program}
The course is structured into eight lectures.
The eight lectures take place during the eight weeks of the course, but are not necessarily in a one-to-one correspondance with the course weeks.

\subsection*{Unit 1}
\paragraph*{Topics}
\begin{itemize}[noitemsep]
	\item Stateful vs stateless computation
  \item Lambda calculus semantics.
    \begin{itemize}[noitemsep]
      \item Variables
      \item Lambda-abstractions/functions
      \item Function application
    \end{itemize}	
  \item Introduction to \textcolor{red}{Typescript}
  \item example of lambda calculus in \textcolor{red}{Typescript}
  \item Bindings and their semantics in lambda-calculus.
  \item Functional \texttt{if-then-else} and differences with its imperative counterpart.
\end{itemize}

\subsection*{Unit 2}
\paragraph*{Topics}			
\begin{itemize}[noitemsep]
	\item Typed lambda calculus. Typing variables, lambda abstractions, function applications.
  \item Recursive functions.
  \item Discriminate unions and records in \textcolor{red}{Typescript}.
  \item Pattern matching
  \item Lists with \texttt{Cons} and \texttt{Empty}
  \item Recursive functions on lists in \textcolor{red}{Typescript}.
\end{itemize}

\subsection*{Unit 3}
\paragraph*{Topics}			
\begin{itemize}[noitemsep]
  \item Higher-order functions.
	\item Pipe operator, function composition, map, fold, map2, fold2.
  \item Curry and Uncurry
  \item Case Study: SQL
\end{itemize}

\subsection*{Unit 4}
\paragraph*{Topics}			
\begin{itemize}[noitemsep]
  \item Immutable trees.
  \item Immutable Binary Search Tree. Find, Add, Remove.
  \item Case Study: Expression evaluation  
\end{itemize}

\subsection*{Unit 5}
\paragraph*{Topics}
\begin{itemize}[noitemsep]
  \item \texttt{map} and \texttt{fold} on Binary Search Trees.
  \item \texttt{map} and \texttt{fold} on \texttt{Option}.  
  \item Functors with records of functions.
\end{itemize}

\newpage
\section{Assessment}
The course is tested with two exams: a written exam and a practical exam. Both parts happen exclusively on paper and must be sufficient in order to pass the course.

\subsection{Theoretical examination \modulecode}
The general shape of an exam for \texttt{\modulecode} is made up of a short series of highly structured open questions.
In each exam the content of the questions will change, but the structure of the questions will remain the same. Questions might include (but not limited to): apply the semantics of lambda calculus on a small function, determine the type of a functional program, determine the result of the execution of a functional program. A sample exam will be provided during the course.

\subsection{Practical examination \modulecode}
The practical exam requires to implement a series of functions in the language F\# described in the exam text.

\subsection{Corona Epidemic:}
Given the current situation the exam cannot happen until the school opens again. We explored ways to have an online examination, including GrandeOmega, but with such short notice we cannot have the course in Grande Omega running properly, due to F\# language not being supported at the moment. We do not know at the moment when written exams can be scheduled again. This affects the students of the older Development 8 course as well, which cannot retake the course until we can have exams again within the school building.

\newpage

\newpage
%\bibliographystyle{plain}
%\bibliography{references}

\printglossaries

\section*{Appendix 1: Assessment matrix}
	\begin{tabular}{|p{2cm}|p{4cm}|}
		\hline
		Learning objective & Dublin descriptors \\
		\hline
		\glssymbol{fpvsimp} & 1, 4, 5 \\
		\hline
        \glssymbol{red} & 1, 2, 4, 5 \\
        \hline
        \glssymbol{typ}& 1, 4, 5 \\
        \hline
        \glssymbol{fpext} & 1, 2 \\
        \hline
	\end{tabular}
	
	\vspace{1cm}

	Dublin-descriptors:
	\begin{enumerate}
		\item Knowledge and understanding
		\item Applying knowledge and understanding
		\item Making judgments
		\item Communication
		\item Learning skills
	\end{enumerate}


%\newpage
%\input{tex/Bijlage2}
%\newpage
%\input{tex/Bijlage3}
\printindex


\end{document}

