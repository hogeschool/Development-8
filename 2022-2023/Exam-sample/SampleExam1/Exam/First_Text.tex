\centering
\Huge
\textbf{Exam procedure}
\flushleft
\normalsize
\examInstruction
\newpage

\centering
\Huge
\textbf{Theoretical part}
\normalsize
\flushleft

\section*{Instructions}
\theoryInstruction

\large
\textbf{Beta-reduction rules:}
\normalsize

\vspace{0.5cm}
\textbf{Variables:}
\varRule

\textbf{Function application:}
\lambdaRule

\textbf{Application}
\applicationRule


\vspace{0.5cm}
\exercise{Given the following untyped lambda-calculus expression:

\begin{equation*}
(\fun x \; y \rightarrow y) \; (\fun x \rightarrow y \; x)
\end{equation*}

replace the requested terms with the elements from the expression in the following lambda-calculus rule that evaluates it:

\lambdaRule

\begin{equation*}
\left \lbrace
\begin{array}{l}
  x = \dots \\
  t = \dots \\
  u = \dots \\
  t[x \mapsto u] = \dots
\end{array} \right.
\end{equation*}
}

\vspace{0.5cm}
\exercise{Given the following untyped lambda-calculus expression:

\begin{equation*}
(\fun x \; y \rightarrow t \; x) \; (((\fun x \; y \rightarrow y \; x) \; (\fun x \rightarrow x)) \; (\fun x \rightarrow x))
\end{equation*}

replace the requested terms with the elements from the expression in the following lambda-calculus rule that evaluates it:

\applicationRule

\begin{equation*}
\left \lbrace
\begin{array}{l}
  t = \dots \\
  u = \dots \\
  t' = \dots \\
  u'= \dots \\
  v = \dots
\end{array} \right.
\end{equation*}
}

\exercise{Complete the missing types (denoted with \texttt{\emptyPlace}) in the following code:

\vspace{0.5cm}
\texttt{let foo (x : int -> string) (y : int) : string = x y\\
let (f : \emptyPlace) = foo(fun (x : int) -> string x)}
}

\vspace{0.5cm}
\exercise{Complete the missing types (denoted with \texttt{\emptyPlace}) in the following code. The dots denote missing code implementation \textbf{that is omitted for brevity and you do not have to complete}:

\vspace{0.5cm}
\texttt{let map (f : 'a -> 'b) -> (l : List<'a>) : List<'b> = ... \\
let (x : \emptyPlace) = map (fun (x : int) -> (string x) + "1")}
}

\vspace{0.5cm}
\exercise{
  Complete the missing types (denoted with \texttt{\emptyPlace}) in the following code. The dots denote missing code implementation \textbf{that is omitted for brevity and you do not have to complete}:

  \vspace{0.5cm}  
  \texttt{let curry (f : 'a * 'b -> 'c) : 'a -> 'b -> 'c = ...\\
    let add (x : int, y : int) : int = ...\\
    let (t : \emptyPlace) = curry add 5
  }
}

\newpage
\centering
\Huge
\textbf{Practical part}
\normalsize
\flushleft

\section*{Instructions}
\practicalInstruction

\setcounter{ExerciseCount}{1}

\vspace{0.5cm}
\exercise{
  \functionEx{rectangle}{(width : int) (height : int) : string}{
      that returns a rectangle of asterisks
      
      \textbf{Example:} calling \texttt{rectangle 4 6} returns \textbf{in the terminal}:\\ 
      \texttt{****\\
        ****\\
        ****\\
        ****\\
        ****\\
        ****\\
      }
    }
}

\vspace{0.5cm}
\exercise{
  \functionEx{suffixes}{(l : List<'a>) : List<List<'a>>}{
    that returns all the possible suffixes of a list. A suffix of a list is a subset of its elements where a number of the elements has been removed from the front of the list.  The whole list is also considered a suffix of the list itself.
    
    \textbf{Example:}\texttt{ suffixes [5; 2; 1; 5] = [[5; 2; 1; 5]; [2; 1; 5]; [1; 5]; [5]; []]}
}}

\vspace{0.5cm}
\exercise{
  \functionEx{nearBy}{(mapObject : MapObject) (objects : List<MapObject>) (radius : float) : List<MapObject>}{
    For this exercise you can use the method \texttt{Distance} defined in \texttt{Point}. \texttt{MapObject} contains a field of type \texttt{Point}.
    Consider a list of map objects whose type is defined in the template below and a single map object of the same type. The function \texttt{nearBy} finds all map objects within a specified \texttt{radius} containing the same \texttt{Symbol}.
  }
}\\


\vspace{0.5cm}
\exercise{
  \functionEx{maxBy}{(projection : 'a -> 'b) (l : List<'a>) : Option<'b>}{
    that, given a projection function that transforms the element of a list, computes the maximum value among the transformed elements. The function returns \texttt{None} if the list is empty.
    
    \textbf{Example: } \texttt{maxBy (fun (x,y) -> y) [(1,2);(3,4);(-1,-1)] = Some 4}
  }
}\\

\vspace{0.5cm}
\exercise{
  \functionEx{allPaths}{(tree : Tree<'a>) : List<List<'a>>}{
    Given the tree data structure defined in the template below, \texttt{allPaths} compute all possible paths starting from the root of the tree. Remember that the path of a tree is a sequence of adjacent nodes starting from the root and ending up in a leaf. 
  }
}
